% To produce Postscript and PDF:
%    latex template; latex template; 
%    dvips -o template.ps template; ps2pdf template.ps
\documentclass[a4paper,11pt]{article}
\usepackage{mathptmx}
%\usepackage{newtxmath}   not available ? 
%\usepackage{newtxtext} []  not available ?
\usepackage[margin=26mm]{geometry}% <-------- CHANGE HERE for the global margins
\usepackage[T1]{fontenc}    % Check that ÖÄÅöäå come out ok!
\usepackage[utf8]{inputenc}
\usepackage{graphicx}
\usepackage{url} 
\usepackage{parskip}
\usepackage{listings} % for code snippets
\usepackage{authblk}
\usepackage[nobottomtitles]{titlesec}
\linespread{1.15}
%
% should not USE  ? 
%\setlength{\parindent}{0mm} % Do not indent the 1st line of a paragraph.
%\setlength{\parskip}{3mm}   % Add space between paragraphs.

\newenvironment{filecode}[1][]
  {\minipage{\linewidth}% \begin{filecode}[#1]
   \lstset{basicstyle=\ttfamily\footnotesize,#1}}
  {\endminipage}% \end{filecode}

% defines my code style
\usepackage{color}
\usepackage{courier}

\definecolor{dkgreen}{rgb}{0,0.6,0}
\definecolor{gray}{rgb}{0.5,0.5,0.5}
\definecolor{mauve}{rgb}{0.58,0,0.82}

\lstset{frame=tb,
  language=Java,
  aboveskip=3mm,
  belowskip=3mm,
  showstringspaces=false,
  columns=flexible,
  basicstyle={\scriptsize\ttfamily},
  numbers=left,
  numbersep=4pt,
  numberstyle=\scriptsize\color{gray},
  keywordstyle=\color{blue},
  commentstyle=\color{dkgreen},
  stringstyle=\color{mauve},
  moredelim=[s][\color{gray}]{@}{\ },
  breaklines=true,
  breakatwhitespace=true,
  tabsize=2
}

\lstdefinelanguage{JavaScript}{
  keywords={typeof, new, true, false, catch, function, return, null, catch, switch, var, if, in, while, do, else, case, break},
  keywordstyle=\color{blue},
  ndkeywords={class, export, boolean, throw, implements, import, this},
  ndkeywordstyle=\color{gray},
  identifierstyle=\color{black},
  sensitive=false,
  comment=[l]{//},
  morecomment=[s]{/*}{*/},
  commentstyle=\color{dkgreen}\ttfamily,
  stringstyle=\color{blue}\ttfamily,
  morestring=[b]',
  morestring=[b]"
} 

% Save some paper by stuffing more text on each page:
% A4: 210mm x 297mm, approximately 35 mm margins on every side.
\addtolength{\topmargin}{-2mm}    
\addtolength{\textheight}{4mm}    
%\addtolength{\oddsidemargin}{-10mm} 
%\addtolength{\textwidth}{14mm}     

\renewenvironment{abstract}
{\itshape \small
  \begin{center}
  \bfseries \abstractname\vspace{-.5em}\vspace{0pt}
  \end{center}
  \list{}{
    \setlength{\leftmargin}{1.5cm}%
    \setlength{\rightmargin}{\leftmargin}%
  }%
  \item\relax}
{\endlist}

% This creates the header.
\makeatletter
\renewcommand{\@oddhead}
{\fontsize{10}{12}\selectfont \hfill OpenGL ES 3.0 and WebGL 1.0 on Android platform \hfill }
\makeatother

\renewcommand{\Authfont}{\Large\normalfont}
\renewcommand{\Affilfont}{\large\itshape}

\begin{document}

%============================================================
% title
\label{Title} 
\title{OpenGL ES 3.0 and WebGL 1.0 on Android platform \vspace{1pc}}
\author{Patryk Małek \vspace{-0.7pc}}
\affil{
        University of Novi Sad\\
        Faculty of Sciences\\
        malekpatryk@gmail.com
      }
\date{}%\date{\today}         % Do not print the date on the final paper!
\maketitle
% prevents page numbering on this page
\thispagestyle{empty}

%============================================================

\vspace{4pc}
\centerline{
\includegraphics[width=0.35\textwidth,height=0.35\textheight,keepaspectratio]{NoviSadLogoGray.jpg}
}
\vspace{5pc}

\begin{abstract}
\label{Abstract}
Mobile devices are successfully managing to solve tasks primarily aimed for desktop computers.
With their computational power rapidly increasing it's nowadays quite common for 3D games' or applications' developers to add mobile platforms as an aim for their products.  
Speaking of which, Android platform is definately one of the most well known and ubiquitous mobile systems available on the market.
It has gained much popularity in the past few years, spreading from mobile devices to tv sets, automotive industry and multimedia tablets. 
\newline To meet users' growing expectations of multimedia content delivery methods and their demand for new astonishing features new open standards, like HTML5 and WebGL, were introduced. 
The past few years have been very fruitful for developers working 2D and 3D visualisation techiniques for embedded systems and many new framworks and APIs have been developed.
\newline The goal of this paper is to present basic concepts, history of OpenGL ES 3.0 and WebGL 1.0 along with exemplar code snippets for those frameworks/APIs.
Presented code will be targeted for Android platform.
\end{abstract}
\pagebreak

%============================================================

%\tableofcontents

%============================================================

\section{Introduction} 

\subsection{Background}
Nowadays, mobile devices such as cell phones or PDAs have become the most ubiquitous ones of all.
Especially Android platform which is getting more and more popular which each year or even with each month.
In the last months of 2013 Android reached 81\% market share in smartphone market.
With increasing demand for new features, user experiences and ongoing evolution of chips,  developers and designers are working hard to keep up with users' desires using state of the art hardware and sofware.
Smartphones’ and tablets’ high performance graphics processors now easily handle 3D visualization and there is no shortage in memory either.
In the past few years there has been a lot of development in the field of 3D visualisation in embedded systems e.g.\ Android platform which has given life to many frameworks or APIs like OpenGL ES, WebGL, three.js or many others.

\subsection{OpenGL ES}
\emph{OpenGL for Embedded Systems} \cite{opengles_kronos} is a subset of \emph{OpenGL} \cite{opengl_kronos} API for rendering 2D and 3D graphics created by \emph{Kronos Group} \cite{kronos_group} 
It has been designed to be used with embedded systems like smartphones, tablets or video consoles.
OpenGL ES shows a good example of re-constructing a general purpose desktop 3D graphics library to a small, low-level rendering library for embedded systems.
OpenGL ES aimed to provide an extremely compact API without sacrificing features.
Its primary objective (for the release of version 1.0) was to be implementable fully in software in under 50kB of code while being well-suited for hardware acceleration \cite{mobile_3d_graphics_with_OGLES_M3G}.
The graphics effects familiar from desktops are available on mobile devices thanks to packed with features, compressed in size library.
\newline OpenGL ES always provides compatibility with OpenGL's version that it has been based on.
This way it is allowing developers to immediately port their mobile version of application to desktops and then add desktop specific features or effects to their product.
This flexibility of OpenGL ES allows developers for faster and easier porting applications between mobile and desktop devices.
\newline First version that was released in 2001, was version 1.0 based on API from OpenGL 1.3.
It continued most of the functionality from OpenGL and added some as well (for instance, removed need for wraping OpenGL calls with \emph{glBegin} and \emph{glEnd} ).
\newline In 2004 Kronos Group OpenGL 1.1 which improved image quality and optimizations to increase performance while reducing memory bandwidth usage to save power.
\newline In march 2007 OpenGL ES 2.0 has been released and enabled fully programmable 3D graphics (thanks to programmable pipeline).
It was based on OpenGL 2.0 and this compatibility lasted until release of OpenGL 4.1.
\newline The most recent version of OpenGL ES was released in August 2013 and it is marked as 3.0.
It is backwards compatible with version 2.0 and it was based on OpenGL 4.3 API.

\subsection{WebGL}
\emph{WebGL} \cite{webgl_kronos} is a cross platform, web standard JavaScript API for rendering low-level 3D content in any compatible web browser without use of any external plugins like e.g.\ Adobe Flash.
It is exposed through HTML5's Canvas element as DOM (Document Object Model) interface.
Nowadays most of the current versions of web browsers (including the mobile versions) support WebGL.
WebGL grew out of the Canvas 3D experiments started by Vladimir Vukićević at Mozilla and has been presented in 2006 as a Canvas 3D prototype.
By the end of 2007, both Mozilla and Opera had made their own separate implementations.
In early 2009, the non-profit technology consortium Khronos Group started the WebGL Working Group, with initial participation from Apple, Google, Mozilla, Opera, and others.
Version 1.0 of the WebGL specification was released in March 2011.
Early applications of WebGL include Google Maps and Zygote Body \cite{zygote_body}.
More recently, Autodesk has ported most of their applications to the cloud running on local WebGL clients.
Development of the WebGL 2 specification started in 2013 and its specification is based on OpenGL ES 3.0.

\subsection{Android platform}
\emph{Android} \cite{androidcom} is an operating system (based on Linux operating system \cite{gnulinux}), primarily designed for smartphones but after time it has also been developed for tablets, tvs etc.
The very first unofficial versions of Android came out in late 2008 but the first official to be supported by Google came out on April 30th 2009 marked as version 1.5 - \emph{Cupcake}.
\newline Currently is the most popular operating system for mobile devices with around 80\% of market share.
OpenGL ES and Android already have quite a history together.
First version of OpenGL ES API marked as 1.0 was implemented in Android platform in version 1.0 marked as - \emph{Apple Pie}.
After that there has been more improvements and features implemented with each version.
Currently, Android platform since its version 4.3 support the latest OpenGL ES version 3.0.

\subsection{Deployment platform - Google Nexus 4}
For deployment purposes \emph{Google Nexus 4} has been chosen.
Google branded LG smarthone, released on November 13 2013.
\newline Having mid-high end specs: processor Qualcomm Snapdragon™ S4 Pro with 4 cores each 1.5 Ghz and 2GB of RAM it will definately fit purposes as a deployment platform.
\newline With its recent update from Google to Android 4.3 it has received support for OpenGL ES 3.0 together with many software improvement like: enhancements for rendering pipeline, a new version of GLSL ES shading language or greatly enhanced texturing functionalities.

%============================================================

\pagebreak[3]
\section{OpenGL ES \& WebGL}

% reference to section examplae
% (Sec.~\ref{sec:emphasis})
%------------------------------------------------------------

\subsection{OpenGL ES}

\subsubsection{New features in OpenGL ES 3.0}
New functionality provided by OpenGL ES 3.0 specification includes:
\begin{itemize}
\item Instanced Rendering – less draw calls to render the same geometry multiple times (think of big crowd of people where each one of them differs in model matrix and a few appearance attributes (e.g. a texture layer of a texture array) rendered with one call), very good presentation was shown by \emph{PowerVR Graphics} \cite{powervr_graphics} on their Metropolis Demo \cite{powervr_metropolis} with instancing of big sky scrapers,
\item Transform Feedback – stores the primitives generated by the Vertex Processing step(s), recording data from those primitives into Buffer Objects.
This allows one to preserve the post-transform rendering state of an object and resubmit this data multiple times.
\item more internal texture formats (including new compression modes: ETC2/EAC),
\item a new version of the GLSL: ES 300 (shading language) based on GLSL 330 from desktop GL,
\item Multiple Render Targets (MRT) – needed to render to multiple textures at once, e.g. for deferred shading,
\item enhanced texturing functionality,
\item Uniform Buffer Objects – e.g. useful for simpler handling of uniforms shared over multiple programs,
\item and many others not mentioned here for brevity.
\end{itemize}

\subsubsection{Interaction with OpenGL ES on Android}
Most of the API calls using OpenGL ES on Android are the same as for the desktop version of OpenGL. 
To start working with OpenGL ES one can require OpenGL functionality in the manifest file to exclude devices that do not support OpenGL ES in the particular version (it is not a necessary step but it will prevent compatibility errors on devices not supporting it).
Usage of OpenGL ES directive in manifest file is shown on Listing~\ref{lst:opengl_es_manifest}.

\lstinputlisting[label={lst:opengl_es_manifest},caption={Requesting OpenGL ES in Android Manifest file.},language=xml]{./code/manifest_require_opengles.xml}

%\begin{filecode}[label=lst:opengl_es_manifest,caption=Requesting OpenGL ES in Android Manifest file.]
%  \lstinputlisting{./code/manifest_require_opengles.xml}
%\end{filecode}

One can also check what version is supported on the device by using the code snippet from Listing~\ref{lst:check_opengl_es_version}.

\begin{filecode}[label=lst:check_opengl_es_version,caption=Checking OpenGL ES version support on the device.]
  \lstinputlisting{./code/check_opengl_es_version.java}
\end{filecode}

%\lstinputlisting[float=ht,label={lst:check_opengl_es_version},caption={Checking OpenGL ES version support on the device.}]{./code/check_opengl_es_version.java}

Writing an application that uses OpenGL for all or part of its rendering, one would use \emph{GLSurfaceView} (an implementation of \emph{SurfaceView} that uses the dedicated surface for displaying OpenGL rendering) \cite{android_glsurfaceview} as a base for its application.
It is also possible to implement OpenGL applications with \emph{TextureView} (good for partial OpenGL rendering in one's applications) or Android's SurfaceView but it would require a little bit more of additional code.
\newline GLSurfaceView is a specialized \emph{View} container that enables rendering with use of OpenGL calls on devices with Android OS where \emph{GLSurfaceView.Renderer} controls what is being drawn on that view.

GLSurfaceView among many others, supplies the following features:
\begin{itemize}
\item Manages a surface, which is a special piece of memory that can be composited into the Android view system.
\item Manages an EGL display, which enables OpenGL to render into a surface.
\item Accepts a user-provided Renderer object that does the actual rendering.
\item Supports both on-demand and continuous rendering.
\end{itemize}

Below, one can observe a minimal implementation of an \emph{Activity} class \cite{android_activity} that would allow interaction with GLSurfaceView.

\lstinputlisting[caption={Minimal implementation of Android's Activity class that would use GLSurfaceView.}]{./code/openglesactivity.java}

To write an application beyond the basics presented above, one would have to write his own GLSurfaceView.Renderer. The renderer is responsible for making OpenGL calls to render a frame.
Its interface has only three methods to override:

\begin{itemize}
\item \emph{onSurfaceCreated()} which is called at the start of rendering, and whenever the OpenGL ES drawing context has to be recreated (the drawing context is typically lost and recreated when the activity is paused and resumed),
\item \emph{onSurfaceChanged()} method is called when the surface changes size. It's a good place to set your OpenGL viewports, or cameras,
\item \emph{onDrawFrame()} method is called every frame, and it is responsible for drawing the scene. You would typically start by calling glClear to clear the framebuffer, followed by other OpenGL ES calls to draw the current scene's objects, primitives, etc.
\end{itemize}

One of the simplest implementations of GLSurfaceView.Renderer that clears the screen to black color on every frame and does not allocate any resource on Surface creation is presented below:

\lstinputlisting[caption={Implementation of GLSurfaceView.Renderer that clears the screen to black on every frame.}]{./code/minimal_GLSurfaceView_Renderer.java}

Developers using GLSurfaceView should extend this class in order to define how the application should respond to touch events (basic implementation of GLSurfaceView does not cover that). 
\newline 




\pagebreak[3] 
\subsection{WebGL}

%============================================================

\clearpage 
\section{Programming Examples} 

In this section we will try to demonstrate the power of OpenGL ES API 

\lstinputlisting[caption={Exemplar piece of code in Java using OpenGL ES API}]{code/example1.java}

%============================================================

\section{Summary} 

%============================================================

\clearpage
\label{Bibliography} 
%
\bibliographystyle{plain}
\footnotesize{ \bibliography{bibliography} }
% In that case, remember to run bibtex:
% latex template; bibtex template; latex template; latex template; 

\end{document}